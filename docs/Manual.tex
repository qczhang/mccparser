%%%%%%%%%%%%%%%%%%%%%%%%%%%%%% -*- Mode: Latex -*- %%%%%%%%%%%%%%%%%%%%%%%%%%%%
%% Manual.tex
%% Copyright � 2000-01 Laboratoire de Biologie Informatique et Th�orique.
%% Author           : Martin Larose
%% Created On       : Thu Aug 31 12:48:48 2000
%% Last Modified By : Martin Larose
%% Last Modified On : Wed Jul  4 12:02:53 2001
%% Update Count     : 10
%% Status           : Ok.
%%%%%%%%%%%%%%%%%%%%%%%%%%%%%%%%%%%%%%%%%%%%%%%%%%%%%%%%%%%%%%%%%%%%%%%%%%%%%%%


\documentclass[11pt]{article}
\usepackage{makeidx}
\usepackage{times}
\usepackage{isolatin1}
\usepackage[ps2pdf,backref,hyperindex,colorlinks,bookmarks]{hyperref}
\title{{\sc mccparser} manual}
\author{Laboratoire de Biologie Informatique et Th�orique}
\date{\today}

\pagestyle{myheadings}
\markboth{MCCPARSER Manual}{MCCPARSER Manual}

\newcommand{\keyword}[1]{{\rm {\tt <}{\em #1}{\tt >}}}
\newcommand{\nonterminal}[1]{{\rm {\tt <}{\em #1}{\tt >}}}
\newcommand{\terminal}[1]{{\em '}{\bf #1}{\em '}}
\newcommand{\token}[1]{{\bf #1}}
\newenvironment{grammarrule}[2]{\begin{tabbing}
    \keyword{#1} :: \= | \= #2 \kill
    \keyword{#1} :: \> \> {\em #2} \\}
  {\end{tabbing}}
\newenvironment{biggrammarrule}[3]{\begin{tabbing}
    \keyword{#1} :: \= | \= #2 \kill
    \keyword{#1} :: \> \> {\em #2} \\
    \> \> #3 \\}
  {\end{tabbing}}
\newcommand{\altrule}[1]{\> {\tt |} \> {\em #1} \\}
    
\begin{document}
\maketitle

{\parindent 0pt \rule{\textwidth}{.5mm}}

\vspace{20pt}


This manual is intended for internal use.  It describes the general
operation of the {\sc mcsym} script parser ({\sc mccparser}).  The grammar
follows {\sc mcsym} version 3.1 with few changes.  These changes were made
to simplify the grammar implementation.  The mcc grammar is defined in
Appendix~\ref{app:grammar} and the recognized tokens in
Appendix~\ref{app:tokens}.  The parser is distributed as a library and is
able to work on interactive input as well as file input.

\section{How to use the parser}

The installation procedure of the package places {\tt libmccparser.so},
{\tt libmccparser.a} and {\tt mccparser.h} in the chosen {\tt libdir} and
{\tt includedir} respectively.  To get instruction on how to install the
package see the {\tt INSTALL} file in the root directory of the package.

\noindent The following will explain how to use the mcsym script parser.
To include the parser in an application, you must follow these instructions:
\begin{itemize}
\item first include the file {\tt mccparser.h} in your application;
\item assign to the boolean variable {\tt interactive\_parser} the true
  value for an interactive parser or a false value when a file is to be
  parsed;
\item assign to {\tt mccin} the {\tt FILE*} of the input; assign stdin for
  interactive input or the result of the {\tt fopen} call on the input file;
\item when you are using a interactive parser, create a new CMccInput or one
  of it's derivative and assign it to the input\_class variable; in case of
  a batch parser, you don't need to assign it any value;
\item use {\tt mccparse ()} to parse the input.  The result can be found
  in the {\tt astv} variable defined in the library.  Its type is
  \verb+vector< MccPStruc* >*+.  We can display the parsed stream by using
  the method {\tt display (ostream\&)} defined for each {\tt MccPStruc};
\item derive a class from MccVisitor and define the Visit methods that
  evaluates the abstract syntax tree;
\item call {\tt Accept} using your MccVisitor class as parameter each time
  you need to evaluate a node;
\item link your application with the mccparser: {\tt -lmccparser}.
\end{itemize}

\noindent {\bf Example for a simple batch parser}
\begin{verbatim}
eval (FILE *fd)
{
  vector< MccPStruct* >::iterator it;

  mccin = fd;
  interactive_parser = false;
  astv = 0;
  MyVisitor visitor;

  try
    {
      mccparse ();
      for (it = astv->begin (); it != astv->end (); it++)
        (*it)->Accept (&visitor);
    }
  catch (...)
    {
      ...
    }
  if (astv)
    {
      for (it = astv->begin (); it != astv->end (); it++)
        delete *it;
      delete astv;
      astv = 0;
    }

  fclose (mccin);
}
\end{verbatim}


\noindent {\bf Example for a simple interactive parser}
\begin{verbatim}
eval ()
{
  vector< MccPStruct* >::iterator it;

  mccin = stdin;
  input_class = new CMccInput ();
  interactive_parser = true;
  astv = 0;
  MyVisitor visitor;

  while (true)
    {
      try
        {
          mccparse ();
          for (it = astv->begin (); it != astv->end (); ++it)
            (*it)->Accept (&visitor);
        }
      catch (...)
        {
          ...
        }
      if (astv)
        {
          for (it = astv->begin (); it != astv->end (); ++it)
            delete *it;
          delete astv;
        }
    }
}
\end{verbatim}


\noindent {\bf Example for multifile non-interactive parser}
\begin{verbatim}
eval (FILE *fd)
{
  vector< MccPStruct* > *last_astv = astv;
  bool last_interactive = interactive_parser;
  FILE *last_mccin = mccin;
  vector< MccPStruct* >::iterator it;
  MyVisitor visitor;

  mccin = fd;
  interactive_parser = false;
  astv = 0;
  try
    {
      mccparse ();
      for (it = astv->begin (); it != astv->end (); it++)
        (*it)->Accept (&visitor);
    }
  catch (...)
    {
      ...
    }
  if (astv)
    {
      for (it = astv->begin (); it != astv->end (); it++)
        delete *it;
      delete astv;
      astv = 0;
    }

  fclose (mccin);
  astv = last_astv;
  interactive_parser = last_interactive;
  // mccin is reassign and repositionned using this procedure.
  mccrestart (last_mccin);
}
\end{verbatim}

The class CMccInput is a wrapper around the interactive read function.  The
parser calls {\tt input\_class->nextchar ()} to get the next typed
character.  The implementation of CMccInput's nextchar is a simple {\tt getc
  ()} call.  If you want to implement a different input procedure (buffered,
string buffered, etc) you can build a child class of CMccInput and modify
it's {\tt nextchar ()} method.

To read from a string instead of a file, you can follow these instructions:
\begin{itemize}
\item create a {\tt YY\_BUFFER\_STATE} handle with {\tt mcc\_scan\_string
    ()};
\item switch the input from {\tt mccin} to the handle by {\tt
    mcc\_switch\_buffer ()};
\item set {\tt interactive\_buffer} to {\tt false};
\item call {\tt mccparse ()};
\item delete the string handle with {\tt mcc\_delete\_buffer ()}.
\end{itemize}

\noindent {\bf Example for non-interactive string parser}
\begin{verbatim}
void
eval (const char *str)
{
  vector< MccPStruct* > *last_astv = astv;
  bool last_interactive = interactive_parser;
  FILE *last_mccin = mccin;
  MyVisitor visitor;

  try
    {
      vector< MccPStruct* >::iterator it;
      YY_BUFFER_STATE handle = mcc_scan_string (a);
      
      mcc_switch_to_buffer (handle);
      interactive_parser = false;
      astv = 0;

      mccparse ();
      
      for (it = astv->begin (); it != astv->end (); ++it)
        (*it)->Accept (&visitor);

      mcc_delete_buffer (handle);
      for (it = astv->begin (); it != astv->end (); ++it)
        delete *it;
      delete astv;
    }
  catch (...)
    {
      ...
    }
  astv = last_astv;
  interactive_parser = last_interactive;
  // mccin is reassign and repositionned using this procedure.
  mccrestart (last_mccin);
}
\end{verbatim}


\section{Changes}

Changes from {\sc mcsym} version 3.2.4:
\begin{itemize}
\item the MccVisitor class was added;
\item residue, connect, pair statements must contain at least one
  definition;
\item angle, cycle, distance and torsion statements must contain at least
  one definition;
\item fixed\_\,distance, vdw\_\,distance atomset and atomsetopt options must
  be in order in the res\_\,clash statement;
\item align, rmsd\_\,bound, atomset and atomsetopt options must be in order
  in the cache statement;
\end{itemize}

\appendix
\newpage
\section{Grammar definition}
\label{app:grammar}

\begin{grammarrule}{statement\_star}{\nonterminal{statement}*}
\end{grammarrule}
\begin{grammarrule}{statement}{\terminal{sequence} \terminal{(} \token{IDENT}
    \nonterminal{residueRef} \token{IDENT}+ \terminal{)}}
  \altrule{\token{IDENT} \terminal{=} \nonterminal{fgexp}}
  \altrule{\terminal{residue} \terminal{(} \nonterminal{resdef}+ \terminal{)}}
  \altrule{\terminal{connect} \terminal{(} \nonterminal{condef}+ \terminal{)}}
  \altrule{\terminal{pair} \terminal{(} \nonterminal{pairdef}+ \terminal{)}}
  \altrule{\terminal{explore} \terminal{(} \nonterminal{fgRef}
    \nonterminal{expfile}? \terminal{)}}
  \altrule{\terminal{restore} \terminal{(} \token{STRING}
  \nonterminal{expfile}? \terminal{)}}
  \altrule{\terminal{source} \terminal{(} \token{STRING} \terminal{)}}
  \altrule{\terminal{adjacency} \terminal{(} \nonterminal{fgRef}
  \token{FLOAT} \token{FLOAT} \terminal{)}}
  \altrule{\terminal{angle} \terminal{(} \nonterminal{angledef}+ \terminal{)}}
  \altrule{\terminal{res\_\,clash} \terminal{(} \nonterminal{fgRef}
    \nonterminal{fixorvdw}? \token{FLOAT} \nonterminal{atomset}?
    \terminal{no\_\,hydrogen}? \terminal{)}}
  \altrule{\terminal{cycle} \terminal{(} \nonterminal{cycledef}+ \terminal{)}}
  \altrule{\terminal{distance} \terminal{(} \nonterminal{distdef}+ \terminal{)}}
  \altrule{\terminal{torsion} \terminal{(} \nonterminal{torsiondef}+
    \terminal{)}}
  \altrule{\terminal{new\_\,tag} \terminal{(} \nonterminal{resortfo}
    \token{STRING} \nonterminal{queryexp}+ \terminal{)}}
  \altrule{\terminal{add\_\,pdb} \terminal{(} \nonterminal{addpdbdefs}+
    \terminal{)}}
  \altrule{\terminal{display\_\,fg} \terminal{(} \nonterminal{fgRef}
    \terminal{)}}
  \altrule{\terminal{note} \terminal{(} \token{STRING} \terminal{)}}
  \altrule{\terminal{notes}}
  \altrule{\terminal{remark} \terminal{(} \token{STRING} \terminal{)}}
  \altrule{\terminal{reset}}
  \altrule{\terminal{reset\_\,db}}
  \altrule{\terminal{version}}
  \altrule{\terminal{quit}}
\end{grammarrule}
\begin{grammarrule}{resdef}{\nonterminal{residueRef} \nonterminal{residueRef}?
    \nonterminal{queryexp} \token{INTEGER}}
\end{grammarrule}
\begin{grammarrule}{condef}{\nonterminal{residueRef} \nonterminal{residueRef}
    \nonterminal{queryexp} \token{INTEGER}}
  \altrule{\nonterminal{residueRef} \nonterminal{queryexp} \nonterminal{residueRef}
    \nonterminal{queryexp} \nonterminal{queryexp} \token{INTEGER}}
\end{grammarrule}
\begin{grammarrule}{pairdef}{\nonterminal{residueRef} \nonterminal{residueRef}
    \nonterminal{queryexp} \token{INTEGER}}
  \altrule{\nonterminal{residueRef} \nonterminal{queryexp} \nonterminal{residueRef}
    \nonterminal{queryexp} \nonterminal{queryexp} \token{INTEGER}}
\end{grammarrule}
\begin{grammarrule}{expfile}{\terminal{fileName\_\,pdb} \terminal{(}
    \token{STRING} \terminal{)} \terminal{zipped}?}
\end{grammarrule}
\begin{biggrammarrule}{angledef}
  {\nonterminal{residueRef} \terminal{:} \nonterminal{atomRef}
    \nonterminal{residueRef} \terminal{:} \nonterminal{atomRef}}
  {\nonterminal{residueRef} \terminal{:} \nonterminal{atomRef} \token{FLOAT}
    \token{FLOAT}}
\end{biggrammarrule}
\begin{grammarrule}{fixorvdw}{\terminal{fixed\_\,distance}}
  \altrule{\terminal{vdw\_\,distance}}
\end{grammarrule}
\begin{grammarrule}{cycledef}{\nonterminal{residueRef} \nonterminal{residueRef}
    \token{INTEGER} \token{FLOAT}}
\end{grammarrule}
\begin{biggrammarrule}{distdef}
  {\nonterminal{residueRef} \terminal{:} \nonterminal{atomRef}
    \nonterminal{residueRef} \terminal{:} \nonterminal{atomRef}}
  {\token{FLOAT} \token{FLOAT}}
\end{biggrammarrule}
\begin{biggrammarrule}{torsiondef}
  {\nonterminal{residueRef} \terminal{:} \nonterminal{atomRef}
    \nonterminal{residueRef} \terminal{:} \nonterminal{atomRef}}
  {\nonterminal{residueRef} \terminal{:} \nonterminal{atomRef}
    \nonterminal{residueRef} \terminal{:} \nonterminal{atomRef}
    \token{FLOAT} \token{FLOAT}}
\end{biggrammarrule}
\begin{grammarrule}{resortfo}{\terminal{res}}
  \altrule{\terminal{tfo}}
\end{grammarrule}
\begin{grammarrule}{addpdbdefs}{\nonterminal{cutoff}? \token{STRING}+}
\end{grammarrule}
\begin{grammarrule}{cutoff}{\nonterminal{tfo} \nonterminal{cfo}?}
  \altrule{\nonterminal{cfo} \nonterminal{tfo}?}
\end{grammarrule}
\begin{grammarrule}{tfo}{\terminal{tfo\_\,cutoff} \token{FLOAT}}
\end{grammarrule}
\begin{grammarrule}{cfo}{\terminal{confo\_\,cutoff} \token{FLOAT}}
\end{grammarrule}
\begin{grammarrule}{queryexp}{\terminal{\{} \nonterminal{filename}*
    \nonterminal{queryorexp}? \terminal{\}}}
\end{grammarrule}
\begin{grammarrule}{filename}{\terminal{file} \terminal{(} \token{STRING}
    \terminal{)}}
\end{grammarrule}
\begin{grammarrule}{queryorexp}{\nonterminal{queryandexp}}
  \altrule{\nonterminal{queryorexp} \terminal{\boldmath{$||$}}
    \nonterminal{queryandexp}}
\end{grammarrule}
\begin{grammarrule}{queryandexp}{\nonterminal{queryfaceexp}}
  \altrule{\nonterminal{queryandexp} \terminal{\&\&}
    \nonterminal{queryfaceexp}}
\end{grammarrule}
\begin{grammarrule}{queryfaceexp}{\nonterminal{querynotexp}}
  \altrule{\nonterminal{queryidentexp} \terminal{/}
    \nonterminal{queryidentexp}}
\end{grammarrule}
\begin{grammarrule}{querynotexp}{\nonterminal{queryidentexp}}
  \altrule{\terminal{!} \nonterminal{querynotexp}}
\end{grammarrule}
\begin{grammarrule}{queryidentexp}{\token{IDENT}}
  \altrule{\terminal{any}}
  \altrule{\terminal{(} \nonterminal{queryorexp} \terminal{)}}
\end{grammarrule}
\begin{grammarrule}{fgexp}{\terminal{backtrack} \terminal{(} \nonterminal{fgRef}?
    \nonterminal{res\_\,place}+ \terminal{)}}
  \altrule{\terminal{cache} \terminal{(} \nonterminal{fgRef}
    \terminal{align}? \nonterminal{rmsd}? \nonterminal{atomset}?
    \terminal{no\_\,hydrogen}? \terminal{)}}
  \altrule{\terminal{library} \terminal{(} \terminal{pdb} \token{STRING}
    \nonterminal{libopt}* \terminal{)}}
\end{grammarrule}
\begin{grammarrule}{res\_\,place}{\terminal{(} \nonterminal{residueRef}
    \nonterminal{residueRef}* \terminal{)}}
  \altrule{\terminal{place} \terminal{(} \nonterminal{residueRef}
    \nonterminal{residueRef} \nonterminal{fgRef} \terminal{)}}
\end{grammarrule}
\begin{grammarrule}{rmsd}{\terminal{rmsd\_\,bound} \token{FLOAT}}
\end{grammarrule}
\begin{grammarrule}{libopt}{\terminal{strip} \terminal{(}
    \nonterminal{residueRef}+ \terminal{)}}
  \altrule{\terminal{change\_\,id} \terminal{(} \token{STRING} \terminal{,}
    \token{STRING} \terminal{)}}
\end{grammarrule}
\begin{grammarrule}{atomset}{\terminal{all}}
  \altrule{\terminal{base\_\,only}}
  \altrule{\terminal{backbone\_\,only}}
  \altrule{\terminal{pse\_\,only}}
\end{grammarrule}
\begin{grammarrule}{residueRef}{\token{INTEGER}}
  \altrule{\token{RESNAME}}
\end{grammarrule}
\begin{grammarrule}{atomRef}{\token{IDENT}}
  \altrule{\token{RESNAME}}
\end{grammarrule}
\begin{grammarrule}{fgRef}{\token{IDENT}}
  \altrule{\token{QUOTED\_\,IDENT}}
\end{grammarrule}


\newpage
\section{Token definition}
\label{app:tokens}

\begin{tabbing}
  QUOTED\_\,IDENT \= = \/ \= \kill
  INTEGER \> = \> \verb|-?[0-9]+| \\
  FLOAT \> = \> \verb|(-?[0-9]+)(.[0-9]*)?| \\
  RESNAME \> = \> \verb|[a-zA-Z][0-9]+| \\
  IDENT \> = \> \verb|[-_<>a-zA-Z][-_<>\'%$\.a-zA-Z0-9]*| \\
  STRING \> = \> \verb|\"[^\\\"]*\"| \\
  QUOTED\_\,IDENT \> = \> \verb|\'[^\']*\'| \\
\end{tabbing}

\end{document}
